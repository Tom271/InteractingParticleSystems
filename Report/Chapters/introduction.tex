Many natural phenomena arise from the interaction of many agents. Very complex behaviour can emerge from simple interaction rules -- a classic example is the murmurations seen in flocks of starlings. Here wonderful patterns emerge even though each starling interacts with only its 6-7 nearest neighbours \cite{Ballerini08}. Shoaling fish also exhibit similar flocking behaviour. This behaviour provides a social group for the species, as well as providing increased protection from predators and increased chance of mating. It also allows mass migration to occur. Understanding and developing models for these systems is an area of great interest, interacting particles are present from the microscopic scale (mammal cells, bacteria, even electromagnetism), all the way to large mammals communicating across oceans. The goal is to approximate the complexity present in these systems using as few simple rules as possible.

Broadly speaking, there are two closely-linked approaches to modelling particle systems \cite{Berdahl17}. The first is a particle- or agent-based model, in which every agent is modelled as a discrete object updating its behaviour based on other agents in the system. The second is a kinetic model, commonly seen in statistical mechanics for large numbers of molecules. Here there are no particles, we instead consider the density of particles at a given time and configuration of the system. One of the classic studies into collective motion was the work of Vicsek et al. \cite{Vicsek95} in which what is now known as the Vicsek model was proposed.

\subsection{Vicsek Model}
In the Vicsek model, particles align based on the average direction of the velocities all the other particles in the system. That is, at time $t+\Dt$ the position, $x^{i,N}$ of the $i^{\text{th}}$ particle is given by
\[ x^{i,N}(t+\Dt) =     x^{i,N}(t) + v^{i,N}(t)\Dt\]
where $v^{i,N}(t)$ is the velocity of the particle at time $t$. All particles have the same magnitude of velocity, and the direction is given by an angle $\theta$ where
\[\theta(t+\Dt) = \langle \theta(t)\rangle_r + \Delta \theta.\]
Here, $\langle \theta(t)\rangle_r$ denotes the average direction of the velocities of all particles within a distance $r$ of the $i^{\text{th}}$ particle. The term $\Delta \theta$ is a random variable uniformly distributed on $[-\eta/2,\eta/2]$.

This simple system was shown to have a critical point at which the motion of the particles changed from unordered to flocking behaviour, before transitioning to completely ordered motion. 

The original Vicsek model is limited. It cannot incorporate different types of interaction between the particles and the interaction suddenly disappears as a particle moves beyond a distance $r$ from another particle. Other models, such as the Cucker-Smale model \cite{Cucker07}, adapt the model so that the magnitude of the velocity is not fixed, and the interaction is a much smoother function. Both the Vicsek and Cucker-Smale models are very theoretical in their approach. A more traditional biological approach is presented by Couzin et al. \cite{Couzin02}. Here they considered three different regions around each particle: the zones of repulsion, orientation and attraction. The goal here was to create a more natural model that incorporates animals desire to maintain a fixed distance between each other -- preventing collisions or dispersion. A 'field of perception' can also be incorporated, reflecting that most biological agents do not have $360^{\circ}$ vision.

In this report we focus on the model introduced in \cite{Butta2019}. This model is similar to the Vicsek model in that particles align based on other particles velocities, however it is much more flexible in how the interaction is described. In Section \ref{sec:model} the particle model, along with its associated kinetic model, is introduced and described term by term. In Section \ref{sec:dynamics}, the dynamics of the model are investigated. In Section \ref{sec:numericalmethods}, the numerical methods required to approximate these models numerically are presented, before being applied in Section \ref{sec:application}. Avenues for possible future work are given in Section \ref{sec:discussion}.
